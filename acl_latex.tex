\documentclass[11pt]{article}

% Change "review" to "final" to generate the final (sometimes called camera-ready) version.
% Change to "preprint" to generate a non-anonymous version with page numbers.
\usepackage[final]{acl}

% Standard package includes
\usepackage{times}
\usepackage{latexsym}

% For proper rendering and hyphenation of words containing Latin characters (including in bib files)
\usepackage[T1]{fontenc}
% For Vietnamese characters
% \usepackage[T5]{fontenc}
% See https://www.latex-project.org/help/documentation/encguide.pdf for other character sets

% This assumes your files are encoded as UTF8
\usepackage[utf8]{inputenc}

% This is not strictly necessary, and may be commented out,
% but it will improve the layout of the manuscript,
% and will typically save some space.
\usepackage{microtype}

% This is also not strictly necessary, and may be commented out.
% However, it will improve the aesthetics of text in
% the typewriter font.
\usepackage{inconsolata}

%Including images in your LaTeX document requires adding
%additional package(s)
\usepackage{graphicx,amsmath,amssymb,booktabs,array,subcaption,multirow}

% If the title and author information does not fit in the area allocated, uncomment the following
%
%\setlength\titlebox{<dim>}
%
% and set <dim> to something 5cm or larger.

\usepackage[most]{tcolorbox}
\usepackage{xcolor}
\usepackage{listings}

% 定义配色
\definecolor{promptgray}{RGB}{245, 245, 245}
\definecolor{textblue}{RGB}{240, 245, 255}
\definecolor{framegray}{RGB}{200, 200, 200}

% 基础代码样式配置
\lstset{
    basicstyle=\small\ttfamily,
    breaklines=true,               % 允许自动换行
    breakatwhitespace=false,       % 允许在非空格处换行
    columns=flexible,
    keepspaces=true,
    xleftmargin=0pt,
}

\newtcolorbox{rolloutbox}[1]{
    colback=white,
    colframe=framegray,
    fonttitle=\bfseries\ttfamily,
    title=#1,
    enhanced,
    breakable,
    attach title to upper,
    after title={\medskip\hrule\medskip},
    coltitle=black,
    left=5pt,
    right=5pt,
    top=5pt,
    bottom=5pt,
    boxrule=0.5pt,
    arc=3pt
}

% 使用 tcblisting 包装,解决换行溢出问题
\newtcolorbox{promptpart}{
    colback=promptgray,
    boxrule=0pt,
    frame hidden,
    left=2pt, right=2pt, top=2pt, bottom=2pt,
    fontupper=\small\bfseries,
    title=Prompt:,
    fonttitle=\small\bfseries,
    coltitle=black!70,
    attach title to upper={\\},
    after upper=
}

\newtcolorbox{genpart}{
    colback=textblue,
    boxrule=0pt,
    frame hidden,
    left=2pt, right=2pt, top=2pt, bottom=2pt,
    fontupper=\small\bfseries,
    title=Generated Text:,
    fonttitle=\small\bfseries,
    coltitle=black!70,
    attach title to upper={\\},
    after upper=
}

\title{nanoRFT: An Efficient On-Policy Inference Engine for DLMs}

% Author information can be set in various styles:
% For several authors from the same institution:
% \author{Author 1 \and ... \and Author n \\
%         Address line \\ ... \\ Address line}
% if the names do not fit well on one line use
%         Author 1 \\ {\bf Author 2} \\ ... \\ {\bf Author n} \\
% For authors from different institutions:
% \author{Author 1 \\ Address line \\  ... \\ Address line
%         \And  ... \And
%         Author n \\ Address line \\ ... \\ Address line}
% To start a separate ``row'' of authors use \AND, as in
% \author{Author 1 \\ Address line \\  ... \\ Address line
%         \AND
%         Author 2 \\ Address line \\ ... \\ Address line \And
%         Author 3 \\ Address line \\ ... \\ Address line}

\author{
  Linkang Dong \\
  \texttt{253108120084} \\\And
  Jiaxin Wan \\
  \texttt{253108030048} \\\And
  Kangyu Wang \\
  \texttt{253108100066} \\\And
  Bohao Lyu \\
  \texttt{253108120108} \\}

%\author{
%  \textbf{First Author\textsuperscript{1}},
%  \textbf{Second Author\textsuperscript{1,2}},
%  \textbf{Third T. Author\textsuperscript{1}},
%  \textbf{Fourth Author\textsuperscript{1}},
%\\
%  \textbf{Fifth Author\textsuperscript{1,2}},
%  \textbf{Sixth Author\textsuperscript{1}},
%  \textbf{Seventh Author\textsuperscript{1}},
%  \textbf{Eighth Author \textsuperscript{1,2,3,4}},
%\\
%  \textbf{Ninth Author\textsuperscript{1}},
%  \textbf{Tenth Author\textsuperscript{1}},
%  \textbf{Eleventh E. Author\textsuperscript{1,2,3,4,5}},
%  \textbf{Twelfth Author\textsuperscript{1}},
%\\
%  \textbf{Thirteenth Author\textsuperscript{3}},
%  \textbf{Fourteenth F. Author\textsuperscript{2,4}},
%  \textbf{Fifteenth Author\textsuperscript{1}},
%  \textbf{Sixteenth Author\textsuperscript{1}},
%\\
%  \textbf{Seventeenth S. Author\textsuperscript{4,5}},
%  \textbf{Eighteenth Author\textsuperscript{3,4}},
%  \textbf{Nineteenth N. Author\textsuperscript{2,5}},
%  \textbf{Twentieth Author\textsuperscript{1}}
%\\
%\\
%  \textsuperscript{1}Affiliation 1,
%  \textsuperscript{2}Affiliation 2,
%  \textsuperscript{3}Affiliation 3,
%  \textsuperscript{4}Affiliation 4,
%  \textsuperscript{5}Affiliation 5
%\\
%  \small{
%    \textbf{Correspondence:} \href{mailto:email@domain}{email@domain}
%  }
%}

\begin{document}

\maketitle

\begin{abstract}
Diffusion Language Models (DLMs) have emerged as promising alternatives to autoregressive (AR) models, offering unique advantages in parallel generation and complex reasoning. However, their post-training efficiency, particularly via on-policy Reinforcement Learning (RL), is severely constrained by the system-level "Reload-and-Rollout" bottleneck. Existing pipelines rely on disparate training and inference processes coupled via slow file-system I/O, causing significant GPU underutilization. To address this, we introduce \textbf{nanoRFT}, a lightweight and unified framework implementing System-Algorithm Co-design. nanoRFT features a "Twin-Engine" inference backend optimized for both AR (e.g., Qwen) and Diffusion (e.g., LLaDA, SDAR) architectures. By replacing disk-based checkpointing with an NCCL-based real-time parameter broadcast mechanism, we reduce model synchronization latency from minutes to the sub-second level (\textbf{10$\sim$50ms} for LLMs/DLMs). Experimental results on the Countdown reasoning task demonstrate that our system sustains a rollout throughput of \textbf{19,444 tokens/sec} on NVIDIA H200 GPUs—a \textbf{21.76\% improvement} over the vLLM baseline. Furthermore, our specialized kernels for DLMs achieve up to \textbf{1.46$\times$ speedup} compared to existing DLMs engines like JetEngine. We validate the framework by supporting stable on-policy RL algorithms (GRPO) with negligible overhead, providing a scalable baseline for the efficient alignment of next-generation generative models.
\end{abstract}

\section{Introduction}

In the rapidly evolving landscape of Generative AI, \textbf{Diffusion Language Models (DLMs)}---such as LLaDA and SDAR~\citep{JetAstra2025}---have emerged as compelling alternatives to standard Autoregressive (AR) models. By leveraging iterative denoising processes rather than strict left-to-right generation, DLMs enable bidirectional context awareness and parallel decoding, offering distinct advantages in complex reasoning and planning tasks.

While the pre-training of DLMs has proven successful, the \textbf{post-training} phase---specifically Reinforcement Learning (RL) fine-tuning---remains a critical bottleneck. Recent research, such as \textit{Retaining by Doing}~\citep{chen2025retaining}, highlights that \textbf{on-policy RL} significantly outperforms Supervised Fine-Tuning (SFT) by exhibiting ``mode-seeking'' behavior. This capability allows models to align with complex reward functions while mitigating catastrophic forgetting, a crucial requirement for maintaining general capabilities during domain adaptation. However, implementing efficient on-policy RL for DLMs presents severe system-level challenges that current infrastructure fails to address.

\subsection*{1.1 The System-Algorithm Mismatch}
The core conflict lies between the dynamic nature of on-policy training and the static design of existing high-performance inference engines.

\begin{enumerate}
    \item \textbf{The ``Reload-and-Rollout'' Bottleneck}: On-policy algorithms require the inference engine to generate data (rollouts) using the most recently updated model parameters at every step. Mainstream inference frameworks like LMDeploy~\citep{lmdeploy2023} or JetEngine~\citep{JetAstra2025} are optimized for \textbf{static serving}---where weights remain constant. Utilizing them in a training loop forces a prohibitive cycle of saving weights to disk, restarting the engine, and reloading the model. As noted in prior work like \textit{DiRL}~\citep{zhu2025dirl} or DLLM-RL ~\citep{wang2025tracerl}, this I/O latency often exceeds the computation time of the training step itself, leading to massive GPU underutilization.
    
    \item \textbf{Architectural Fragmentation}: The DLM ecosystem lacks unified support. Distinct architectures require specialized handling---such as Blockwise Diffusion for SDAR or Masked Diffusion for LLaDA. Current frameworks often support either AR or DLM, but rarely both, forcing researchers to maintain disparate, unoptimized codebases for hybrid experiments.
\end{enumerate}

\subsection*{1.2 The Proposed Solution: nanoRFT}
To bridge this gap between algorithmic requirements and system capabilities, we introduce \textbf{nanoRFT (nano Reinforcement Learning Fine-Tuning)}. Developed as a lightweight, high-performance framework, nanoRFT implements a \textbf{System-Algorithm Co-design} that unifies training and inference into a coherent loop.

Our approach addresses the aforementioned challenges through three key technical contributions:
\begin{itemize}
    \item \textbf{Twin-Engine Architecture}: We provide a unified interface that supports both standard AR models (via \texttt{nanovllm} for Qwen) and specialized DLMs (via \texttt{nanovdlm} for LLaDA/SDAR). This includes custom Triton kernels for non-causal attention and iterative denoising optimization.
    
    \item \textbf{Sub-Second Weight Synchronization}: To eliminate the I/O bottleneck, we replace file-system operations with an \textbf{NCCL-based broadcast mechanism}. This allows the training engine (HuggingFace-based) to synchronize updated weights to the distributed inference engine in milliseconds rather than minutes.
    
    \item \textbf{Efficient On-Policy Loop}: By integrating \textbf{GRPO} (for AR) and \textbf{DGRPO} (for DLM) directly into the workflow, nanoRFT ensures that data generation is always strictly on-policy without sacrificing throughput, enabling rapid experimentation and stable convergence.
\end{itemize}

This report details the implementation of nanoRFT, demonstrating how optimizing the interaction between the training loop and inference engine unlocks the full potential of Reinforcement Learning for the next generation of language models.

\section{Related Work}

\subsection{High-Performance Inference Engines}

The rapid growth of Large Language Models (LLMs) has driven significant advancements in inference optimization. Several high-performance engines have emerged to address the computational demands of serving these models efficiently.

\textbf{vLLM}~\citep{kwon2023pageattention} introduced \textbf{PagedAttention}, a memory management technique inspired by virtual memory paging in operating systems. By partitioning the KV cache into non-contiguous blocks, vLLM dramatically reduces memory fragmentation and enables efficient batching of requests with varying sequence lengths. This approach has become a foundational technique adopted by many subsequent frameworks.

\textbf{LMDeploy}~\citep{lmdeploy2023} provides an end-to-end deployment solution with optimized kernels for popular model architectures. It supports \textbf{Tensor Parallelism (TP)} for distributing computation across multiple GPUs and implements efficient \textbf{Prefix Caching} to reuse KV cache for shared prompt prefixes, significantly reducing redundant computation in scenarios with common system prompts.

\textbf{JetEngine}~\citep{JetAstra2025} extends optimization techniques specifically for Diffusion Language Models (DLMs). It provides specialized support for iterative denoising processes and blockwise generation strategies required by architectures like SDAR and LLaDA, achieving high throughput through custom CUDA kernels tailored for non-causal attention patterns.

\textbf{nano-vllm}~\citep{nanovllm2025} offers a lightweight alternative that maintains competitive performance while reducing implementation complexity. By focusing on core optimizations---including efficient memory allocation, tensor parallelism, and prefix caching---it provides a flexible foundation suitable for research and rapid prototyping.

\subsection{Limitations in On-Policy Training}

Despite their efficiency in static serving scenarios, these inference engines share a common limitation: they are designed assuming \textbf{fixed model weights}. In on-policy reinforcement learning, where model parameters are updated after each training step, these frameworks require costly checkpoint saving and engine reloading cycles. This fundamental mismatch between dynamic training and static inference motivates our design of nanoRFT, which bridges this gap through real-time weight synchronization.

\section{Methodology}

\label{sec:methodology}

\begin{figure*}[ht]
    \centering
    \includegraphics[width=0.9\textwidth]{figures/nanorft.pdf} 
    \caption{System Architecture of nanoRFT. The Trainer node (GPU 0) performs model updates using HuggingFace, while Inference Workers (GPU 1+) run the optimized inference engines (\texttt{nanovllm}/\texttt{nanovdlm}). Updated weights are broadcasted in real-time via NCCL, enabling efficient on-policy rollouts.}
    \label{fig:arch}
\end{figure*}

To address the system efficiency bottlenecks inherent in post-training Diffusion Language Models (DLMs), we propose \textbf{nanoRFT}, a lightweight framework implementing a System-Algorithm Co-design. Unlike existing frameworks that treat training and inference as isolated stages coupled via file systems, nanoRFT unifies them into a high-throughput on-policy loop.

\subsection{System Architecture: Decoupled Trainer-Inferencer}
Standard reinforcement learning (RL) pipelines for LLMs typically suffer from the "Reload-and-Rollout" latency, where the inference engine must reload updated weights from the disk at every step. We address this by adopting a decoupled architecture:

\begin{itemize}
    \item \textbf{Trainer Node:} Hosted on GPU 0, this module utilizes the HuggingFace library to perform gradient updates (e.g., GRPO/DGRPO). It maintains the master copy of the model parameters.
    \item \textbf{Inference Workers:} Hosted on GPU 1+, these workers run our optimized inference engines. They are responsible for high-throughput rollout generation.
\end{itemize}

Crucially, instead of communicating via disk checkpoints, we implement a \textbf{Real-Time Parameter Broadcast} mechanism based on NCCL. As illustrated in Figure~\ref{fig:arch}, immediately after the optimizer step, the updated weights are broadcasted from the Trainer to Inference Workers via GPU-direct communication protocols. This design reduces synchronization latency from minutes to the sub-second level ($<$30ms for AR models, $\sim$10ms for DLMs), ensuring that data generation remains strictly on-policy without stalling the training loop.

\subsection{Twin-Engine Inference Backend}
To resolve the fragmentation in model support, we introduce a "Twin-Engine" design that provides native optimizations for both Autoregressive (AR) and Diffusion paradigms.

\subsubsection{nanovllm: Optimized AR Inference}
For standard AR models (e.g., Qwen2.5), we employ \texttt{nanovllm}. It integrates state-of-the-art memory management techniques, including \textbf{PagedAttention} and \textbf{Block Manager} ~\citep{kwon2023pageattention}, to handle the dynamic memory footprint of variable-length generation. This engine supports tensor parallelism and prefix caching, ensuring high throughput during the rollout phase of RL training [8].

\subsubsection{nanovdlm: Specialized DLM Adaptation}
Diffusion models require fundamentally different computation patterns, such as iterative denoising and non-causal attention [9, 10]. We developed \texttt{nanovdlm} to specifically address these requirements:
\begin{itemize}
    \item \textbf{Iterative Decoding Support:} Unlike the single-pass generation of AR, \texttt{nanovdlm} supports flexible denoising schedules (e.g., dynamic remasking strategies) required by models like LLaDA\citep{nie2025llada} and SDAR \citep{JetAstra2025}.
    \item \textbf{Kernel Optimization:} We implemented custom high-performance \textbf{Triton kernels} to optimize the bidirectional (non-causal) attention mechanisms inherent in DLMs. This ensures that the compute-intensive denoising steps fully utilize the hardware capabilities of NVIDIA GPUs.
    \item \textbf{Unified Interface:} Despite the architectural differences, \texttt{nanovdlm} exposes a unified \texttt{rollout} interface compatible with our RL algorithms, abstracting away the complexity of block-wise or full-sequence diffusion generation.
\end{itemize}

\subsection{On-Policy Reinforcement Learning Algorithms}
To maintain focus on our system-level contributions and the optimization of the training pipeline, we omit the formal mathematical derivations of the RL objectives, which are adopted from existing literature. Our framework integrates Group Relative Policy Optimization (GRPO), which has proven effective for reasoning tasks~\citep{guo2025deepseekr1}. To accommodate the specific nature of DLMs, we implement two variants:

\begin{enumerate}
    \item \textbf{Standard GRPO (for AR):} We implement a from-scratch version of GRPO integrated with advanced stability techniques, including DAPO~\citep{yu2025dapo} and Dr.GRPO~\citep{liu2025drgrpo}.
    \item \textbf{DGRPO (for DLM):} For diffusion models, we adapt the objective to align with the iterative generation process. As noted in \textit{DiRL}~\citep{zhu2025dirl}, standard next-token objectives are ill-suited for DLMs. Our DGRPO implementation specifically estimates the trajectory likelihood using Monte Carlo sampling.
\end{enumerate}

By combining real-time weight synchronization with specialized inference kernels, nanoRFT enables an efficient ``Generate-Update-Sync" loop, allowing researchers to explore on-policy RL for DLMs with the same efficiency as standard AR models.

\section{Experiments}

In this section, we evaluate the performance of \texttt{nanoRFT} from both system and algorithmic perspectives. We aim to validate the efficiency of our NCCL-based weight synchronization mechanism and the effectiveness of the on-policy training loop for both Autoregressive (AR) and Diffusion Language Models (DLMs).

\subsection{Experimental Setup}
\label{subsec:setup}

We conducted our primary evaluations on the \textbf{Countdown} task, a mathematical reasoning benchmark requiring logical planning. The experimental environment was configured as follows:

\begin{itemize}
    \item \textbf{Models:} For AR baselines, we utilized \texttt{Qwen2.5-3b} and \texttt{Qwen2.5-Math-1.5B}. For DLM experiments, we employed \texttt{SDAR-8B} and \texttt{LLaDA} to verify architectural compatibility.
    \item \textbf{Hardware Topology:} We adopted a decoupled "1 Trainer + 1 Inferencer" setup. One GPU was dedicated to the HuggingFace-based training loop, while a second GPU hosted the \texttt{nano-vllm}/\texttt{nanovdlm} inference engine.
    \item \textbf{Algorithm:} We implemented Group Relative Policy Optimization (GRPO) from scratch, integrating advanced stability techniques including DAPO~\citep{yu2025dapo}, Dr.GRPO~\citep{liu2025drgrpo}, and GPG~\citep{chu2025gpg}.
\end{itemize}

\subsection{System Efficiency Analysis}

\subsubsection{Weight Synchronization Latency}
A critical bottleneck in on-policy RL is the ``Reload-and-Rollout'' latency, which typically involves a two-stage process: the trainer saves the updated checkpoint to disk, and the inference engine subsequently reloads it. In our benchmark, for existing frameworks, we measured the total time required to save the model parameters to local storage plus the time to load them into the GPU memory. In contrast, \textbf{nanoRFT} bypasses the file system entirely by broadcasting weights via NCCL directly between GPUs.

As shown in Table~\ref{tab:latency}, our NCCL-based memory broadcast mechanism significantly outperforms traditional disk I/O methods. For the AR model (\texttt{Qwen2.5-3b}), we achieved a synchronization latency of \textbf{$<$30ms}, whereas the disk-based approach required over 5 seconds. For the larger DLM architecture (\texttt{SDAR-8B}), despite its increased parameter count, the synchronization was completed in approximately \textbf{10ms} thanks to our optimized memory-mapping strategy. This reduction from seconds to milliseconds ensures that the inference engine consistently serves the most up-to-date policy with negligible downtime, effectively eliminating the synchronization bottleneck.

\begin{table}[ht]
\centering
\footnotesize
\setlength{\tabcolsep}{6pt}
\caption{Synchronization latency comparison: Traditional Disk I/O (Save + Load) vs. nanoRFT (NCCL Broadcast).}
\label{tab:latency}
\begin{tabular}{@{}llc@{}} 
\toprule
\textbf{Architecture} & \textbf{Method} & \textbf{Latency} \\ \midrule
Qwen2.5-1.5B (AR) & Disk I/O (Baseline) & $\ge$3.0s \\
\textbf{Qwen2.5-1.5B (AR)} & \textbf{nanoRFT (Ours)} & \textbf{$\sim$30ms} \\ 
\midrule
Qwen2.5-3B (AR) & Disk I/O (Baseline) & $\ge$5.0s \\
\textbf{Qwen2.5-3B (AR)} & \textbf{nanoRFT (Ours)} & \textbf{$\sim$30ms} \\ \midrule
SDAR-8B-Chat (DLM) & Disk I/O (Baseline) & $\ge$15.0s \\
\textbf{SDAR-8B-Chat (DLM)} & \textbf{nanoRFT (Ours)} & \textbf{$\sim$10ms} \\ \midrule
LLaDA-8B-Instruct (DLM) & Disk I/O (Baseline) & $\ge$18.0s \\
\textbf{LLaDA-8B-Instruct (DLM)} & \textbf{nanoRFT (Ours)} & \textbf{$\sim$10ms} \\ \bottomrule
\end{tabular}
\end{table}

\subsubsection{Inference Throughput}
By leveraging \texttt{nano-vllm}'s native optimizations, including tensor parallelism and prefix caching, \texttt{nanoRFT} maintains high throughput during the rollout phase.

To validate the efficiency of the underlying engine, we reference both reported benchmarks and our own empirical results. As detailed in the official \texttt{nano-vllm} repository~\citep{nanovllm2025}, the framework demonstrates superior efficiency on consumer-grade hardware (RTX 4070 Laptop) compared to standard \texttt{vLLM}. To verify performance in a high-performance computing environment, we conducted independent experiments on an NVIDIA H200 GPU using the \texttt{Qwen3-0.6B} model. As shown in Table~\ref{tab:hardware_throughput}, our measured throughput on the H200 reached \textbf{19,444.55 tps}, representing a \textbf{21.76\% improvement} over the standard \texttt{vLLM} baseline (15,969.38 tps) on the same hardware.

Building on this backend, our specialized kernels for Deep Language Models (DLMs) in \texttt{nanoRFT} significantly outperform the \texttt{JetEngine} framework~\citep{JetAstra2025}. As summarized in Table~\ref{tab:tps_comparison}, \texttt{nanoRFT} achieves consistent speedups across various configurations. Notably, for the \texttt{LLaDA-8B-Instruct} model, \texttt{nanoRFT} achieves its highest throughput of \textbf{1,144.44 tps} at a block size of 32, representing a \textbf{1.30$\times$ speedup}. At a block size of 64, it maintains \textbf{1,100.90 tps}, a \textbf{1.11$\times$ improvement} over \texttt{JetEngine}. Across the SDAR tasks, \texttt{nanoRFT} also maintains a steady performance lead, with speedups ranging from \textbf{1.07$\times$ to 1.46$\times$}, peaking at \textbf{4,807.24 tps} for the SDAR-8B-Chat model.

% --- Table 1: Hardware Benchmark (vLLM vs Nano-vLLM) ---
\begin{table}[ht]
\centering
\footnotesize
\caption{Baseline throughput comparison. RTX 4070 results are reported from the \texttt{nano-vllm} repository~\citep{nanovllm2025}, while H200 results are derived from our internal experiments using \texttt{Qwen3-0.6B} (256 concurrent requests).}
\label{tab:hardware_throughput}
\setlength{\tabcolsep}{5pt}
\begin{tabular}{@{}l l c c r@{}}
\toprule
\textbf{Hardware} & \textbf{Engine} & \textbf{Output Tokens} & \textbf{Time (s)} & \textbf{Throughput} \\
\midrule
\multicolumn{5}{l}{\textit{Reported Results (from repo)}:} \\
\multirow{2}{*}{RTX 4070 Laptop} & vLLM & 133,966 & 98.37 & 1,361.84 tps \\
 & \textbf{Nano-vLLM} & 133,966 & \textbf{93.41} & \textbf{1,434.13 tps} \\
\midrule
\multicolumn{5}{l}{\textit{Our Experiments}:} \\
\multirow{2}{*}{H200} & vLLM & 133,966 & 8.39 & 15,969.38 tps \\
 & \textbf{Nano-vLLM} & 133,966 & \textbf{6.89} & \textbf{19,444.55 tps} \\
\bottomrule
\end{tabular}
\end{table}

% --- Table 2: Optimized JetEngine vs nanoRFT with Speedup Column ---
\begin{table}[ht]
\centering
\footnotesize
\setlength{\tabcolsep}{8pt}
\caption{Throughput (tokens/sec) and speedup comparison between \texttt{JetEngine} and \texttt{nanoRFT}.}
\label{tab:tps_comparison}
\begin{tabular}{@{}l c r r r@{}}
\toprule
\textbf{Model} & \textbf{Block Size} & \textbf{JetEngine} & \textbf{nanoRFT(ours)} & \textbf{Speedup} \\
\midrule
\multirow{3}{*}{LLaDA-8B-Instruct} & 128 & 601.39 & \textbf{831.18} & 1.38$\times$ \\
 & 64  & 992.98 & \textbf{1,100.90} & 1.11$\times$ \\
 & 32  & 879.33 & \textbf{1,144.44} & 1.30$\times$ \\
\midrule
\multirow{3}{*}{SDAR-8B-Chat} & 4   & 4,941.68 & \textbf{5,269.71} & 1.07$\times$ \\
 & 8   & 4,361.23 & \textbf{4,774.59} & 1.09$\times$ \\
 & 16  & 3,291.20 & \textbf{4,807.24} & 1.46$\times$ \\
\bottomrule
\end{tabular}
\end{table}

\subsection{Training Convergence and Reasoning Qualitative Analysis}

We validated the functional correctness and algorithmic effectiveness of our \texttt{nanoRFT} pipeline by monitoring training dynamics on the Countdown reasoning task. The comprehensive training metrics are provided in \textbf{Appendix~\ref{sec:appendix_figs}}, while specific generation samples (rollouts) demonstrating the evolution of reasoning are detailed in \textbf{Appendix~\ref{appendix:rollout_samples}}.

\textbf{Emergence of Reasoning from Base Models.} 
It is worth noting that our AR experiments utilized \texttt{Qwen2.5-Math} and \texttt{Qwen3} base models without any prior Supervised Fine-Tuning (SFT). As shown in the reward curves (Figure~\ref{fig:all_metrics_ar}), the models effectively ``learn to speak'' and reason through pure on-policy RL. In contrast, for DLMs, we utilized \texttt{Instruct} or \texttt{Chat} versions to benchmark against state-of-the-art performance levels.

\textbf{Effectiveness of the RL Pipeline.} 
Our framework demonstrates robust convergence across architectures. Specifically, the \texttt{Qwen2.5-Math-1.5B} model reached a \textbf{85\% training success rate}. A critical observation from Figure~\ref{fig:all_metrics_ar}c is that the \textit{Finished Success Rate}—measuring the accuracy of completed responses—converges rapidly to \textbf{1.0}. This indicates that the model masters the underlying logic early, and the remaining performance gap is primarily due to unfinished generations rather than logical errors.

\textbf{Entropy Collapse in DLMs.} 
Through \texttt{nanoRFT}'s high-throughput rollouts, we identified a key challenge in DLM post-training: \textbf{rapid entropy collapse}. As observed in Figure~\ref{fig:all_metrics_dlm}d, the training entropy for DLMs decreases significantly faster than for AR models. The model quickly converges to a state where rollout samples exhibit minimal variance (often only a few tokens difference), providing a system-level explanation for the exploration difficulties in DLM reinforcement learning.

\textbf{Observations of Latent Reasoning.} 
On smaller models like \texttt{Qwen3-0.6B-Base}, we observed a unique \textit{latent reasoning} phenomenon, as illustrated in the samples in \textbf{Appendix~\ref{appendix:rollout_samples}}. 
\begin{itemize}
    \item \textbf{Early Stage (Step 10):} The model begins with repetitive or structured placeholders like ``itemBuilder'' to organize its output, yet the logic remains rudimentary.
    \item \textbf{Mid-to-Late Stage (Step 300--500):} We observe the emergence of unintelligible token sequences at the beginning of the thought process (e.g., \textit{\texttt{\textbackslash u6218\textbackslash u573a\textbackslash u4e0a Blocks...}} or \textit{\texttt{Tap\textbackslash u0c9b\_H...}}). Despite being unreadable to humans, the model retains these segments as they appear to serve as an internal ``scratchpad'' that aids in reaching the correct \texttt{<answer>}. 
    \item \textbf{Optimization:} As training progresses from Step 300 to 500, these latent reasoning fragments become noticeably more compact. We hypothesize that the model identifies these latent computations as beneficial for maximizing rewards and continues to refine them, whereas even smaller models (e.g., \texttt{Qwen2.5-0.5B}) fail to develop this strategy and suffer from immediate entropy collapse.
\end{itemize}

Overall, these results confirm that \texttt{nanoRFT} successfully facilitates stable on-policy learning, allowing models to develop complex internal reasoning strategies that would be difficult to capture in traditional, high-latency training environments.

\section{Collaboration}

To ensure the successful delivery of \textbf{nanoRFT}, the team adopted a cross-functional collaboration model. Each member contributed equally (25\%) across system design, algorithmic implementation, and experimental validation.

\begin{table}[ht]
\centering
\small
\caption{Team Member Contributions and Responsibilities.}
\label{tab:collaboration}
\begin{tabular}{@{}lcp{10.5cm}@{}}
\toprule
\textbf{Name} & \textbf{Contribution} & \textbf{Key Responsibilities \& Technical Work} \\ \midrule
Linkang Dong & 25\% & \textbf{System Architecture \& Communication:} Architected the NCCL-based real-time weight sync system ($<$30ms); implemented the core infrastructure for real-time parameter updates; implemented the RL training loop from scratch; integrated stability techniques (DAPO, GPG); \\ \addlinespace
Jiaxin Wan & 25\% & \textbf{DLM Backend \& Performance:} Developed the \texttt{nanovdlm} engine; authored custom Triton kernels for non-causal attention; optimized SDAR/LLaDA decoding throughput to exceed JetEngine baselines. Integrated diffusion models into the online RL loop.\\ \addlinespace
Kangyu Wang & 25\% & \textbf{Strategy Design \& Evaluation:} Collaborated on DLM decoding strategies and system testing; Led the comprehensive technical report drafting and documentation, conducted large-scale rollout testing on the Countdown reasoning benchmark. \\ \addlinespace
Bohao Lyu & 25\% & \textbf{Quality Assurance \& Documentation:} Led the systematic evaluation of throughput/latency metrics; conducted the literature survey on DLM efficiency; coordinated the technical report structure and experimental data visualization. \\ \bottomrule
\end{tabular}
\end{table}

\section{Conclusion}

In this work, we introduced \textbf{nanoRFT}, a lightweight and unified framework designed to bridge the gap between high-performance inference and on-policy reinforcement learning for both Autoregressive (AR) and Diffusion Language Models (DLMs). By implementing a system-algorithm co-design approach, we successfully addressed the critical ``Reload-and-Rollout'' bottleneck that typically hinders efficient post-training pipelines.

Our experimental results and system evaluations lead to the following key conclusions:

\begin{itemize}
    \item \textbf{Elimination of I/O Latency:} By replacing traditional disk-based checkpointing with an \textbf{NCCL-based real-time broadcast mechanism}, we reduced model synchronization latency from minutes to the sub-second level (\textbf{10$\sim$50ms}). This ensures that the rollout phase remains strictly on-policy without stalling the training loop or sacrificing GPU utilization.
    
    \item \textbf{Superior Inference Throughput:} The \textbf{Twin-Engine} architecture (\texttt{nanovllm} and \texttt{nanovdlm}) achieves significant performance gains. On NVIDIA H200 GPUs, nanoRFT reached a rollout throughput of \textbf{19,444 tokens/sec}, outperforming the vLLM baseline by \textbf{21.76\%}. Furthermore, our specialized Triton kernels for DLMs provided up to a \textbf{1.46$\times$ speedup} over state-of-the-art engines like JetEngine.
    
    \item \textbf{Algorithmic Robustness:} We demonstrated that nanoRFT effectively supports stable on-policy RL (GRPO and DGRPO) on complex reasoning tasks. The consistent convergence and steady improvement in success rates observed in the Countdown benchmark validate that our system-level optimizations maintain high training fidelity while maximizing throughput.
\end{itemize}

In summary, nanoRFT provides a scalable and efficient baseline for the next generation of generative models. By unifying the training and inference ecosystems through hardware-aware co-design, it empowers researchers to explore advanced alignment strategies for DLMs with the same agility and performance previously reserved for standard autoregressive architectures.

\subsection{Future Work}

While nanoRFT demonstrates significant improvements in training efficiency, several avenues remain for future exploration:

\begin{enumerate}
    \item \textbf{Data Parallelism (DP):} Currently, our setup relies on a single trainer-inferencer topology. Extending the framework to support Data Parallelism across multiple nodes is crucial for scaling to larger foundation models (e.g., 70B+ parameters).
    \item \textbf{Advanced Decoding Strategies:} For DLMs, we plan to investigate more sophisticated decoding methods during the rollout phase to further explore the generation space and improve sample efficiency.
    \item \textbf{Unified Support for Multi-Modalities:} Given the versatility of diffusion models, we aim to extend \texttt{nanovdlm} to support multi-modal tasks beyond text generation.
\end{enumerate}

In summary, nanoRFT provides a robust baseline for the community to explore the frontiers of post-training for next-generation generative models.

\bibliography{custom}

\appendix
\section{Rollout Generation Samples}
\label{appendix:rollout_samples}
This appendix provides samples of the generated text (rollouts) from different models at various stages of the reinforcement learning process.

\subsection{Samples from Qwen3-0.6B-Base (AR)}

\subsubsection{Step 10}
\begin{rolloutbox}{Qwen3-0.6B-Base - Step 10}
\begin{promptpart}
\begin{lstlisting}
<|im_start|>system
You are a helpful assistant. You first think about the reasoning process in your mind and then provide the user with the answer.<|im_end|>
<|im_start|>user
Using the numbers [43, 23, 14], create an equation that equals 52. You can use basic arithmetic operations (+, -, *, /) and each number can only be used once. Show your work in <think> </think> tags. And return the final answer in <answer> </answer> tags, for example <answer> (1 + 2) / 3 </answer>.<|im_end|>
<|im_start|>assistant
Let me solve this step by step.
<think>
\end{lstlisting}
\end{promptpart}

\begin{genpart}
\begin{lstlisting}
First, let's identify the numbers: 43, 23, and 14.
itemBuilder
I will combine these numbers in a way that their sum equals 52.
itemBuilder
Here\u2019s how I plan to achieve this:
itemBuilder
1. Add 43 and 23: 43 + 23 = 66
itemBuilder
2. Subtract 14 from the result: 66 - 14 = 52
itemBuilder
So, the equation that equals 52 is: 43 + 23 - 14 = 52
itemBuilder
Therefore, the final answer is: <answer> (43 + 23) - 14 </answer><|endoftext|>
\end{lstlisting}
\end{genpart}
\end{rolloutbox}

\subsubsection{Step 300}
\begin{rolloutbox}{Qwen3-0.6B-Base - Step 300}
\begin{promptpart}
\begin{lstlisting}
<|im_start|>system
You are a helpful assistant. You first think about the reasoning process in your mind and then provide the user with the answer.<|im_end|>
<|im_start|>user
Using the numbers [68, 56, 82, 29], create an equation that equals 65. You can use basic arithmetic operations (+, -, *, /) and each number can only be used once. Show your work in <think> </think> tags. And return the final answer in <answer> </answer> tags, for example <answer> (1 + 2) / 3 </answer>.<|im_end|>
<|im_start|>assistant
Let me solve this step by step.
<think>
\end{lstlisting}
\end{promptpart}

\begin{genpart}
\begin{lstlisting}
\u6218\u573a\u4e0a Blocks\u5171\u8ba1\\S\u52a8\u751f\u6210\u2bdb\ufffd calor\u52a0\u76d6\u4ea4\u7ed9 amount :
1. We have the numbers: 68, 56, 82, 29.
2. We need to find an equation that equals 65.
3. Let's try different combinations of these numbers.
4. One possible combination is: 82 - 68 + 56 - 29.
5. Calculate inside the parentheses: 82 - 68 = 14.
6. Then, 14 + 56 = 70.
7. Finally, 70 - 29 = 41.
8. This doesn't equal 65.
9. Let's try another combination: 82 - 56 + 68 - 29.
10. Calculate inside the parentheses: 82 - 56 = 26.
11. Then, 26 + 68 = 94.
12. Finally, 94 - 29 = 65.
13. This equals 65.
14. The equation is: 82 - 56 + 68 - 29.
15. Calculate inside the parentheses: 82 - 56 = 26.
16. Then, 26 + 68 = 94.
17. Finally, 94 - 29 = 65.
18. The equation is: 82 - 56 + 68 - 29.
19. The final answer is: <answer> (82 - 56) + (68 - 29) </answer>.<|endoftext|>
\end{lstlisting}
\end{genpart}
\end{rolloutbox}

\subsubsection{Step 500}
\begin{rolloutbox}{Qwen3-0.6B-Base - Step 500}
\begin{promptpart}
\begin{lstlisting}
<|im_start|>system
You are a helpful assistant. You first think about the reasoning process in your mind and then provide the user with the answer.<|im_end|>
<|im_start|>user
Using the numbers [68, 56, 82, 29], create an equation that equals 65. You can use basic arithmetic operations (+, -, *, /) and each number can only be used once. Show your work in <think> </think> tags. And return the final answer in <answer> </answer> tags, for example <answer> (1 + 2) / 3 </answer>.<|im_end|>
<|im_start|>assistant
Let me solve this step by step.
<think>
\end{lstlisting}
\end{promptpart}

\begin{genpart}
\begin{lstlisting}
 Tap\u0c9b_H\ufffd.cookies =>
1. We have the numbers: 68, 56, 82, 29.
2. We need to find an equation that equals 65.
3. Let's try different combinations of these numbers.
4. One possible combination is: 82 - 68 + 56 - 29.
5. Calculate inside the parentheses: 82 - 68 = 14.
6. Then, 14 + 56 = 70.
7. Finally, 70 - 29 = 41.
8. This doesn't equal 65.
9. Let's try another combination: 82 - 56 + 68 - 29.
10. Calculate inside the parentheses: 82 - 56 = 26.
11. Then, 26 + 68 = 94.
12. Finally, 94 - 29 = 65.
13. This equals 65.
14. The equation is: 82 - 56 + 68 - 29.
15. Calculate inside the parentheses: 82 - 56 = 26.
16. Then, 26 + 68 = 94.
17. Finally, 94 - 29 = 65.
18. The equation is: 82 - 56 + 68 - 29.
19. The final answer is: <answer> (82 - 56) + (68 - 29) </answer>.<|endoftext|>
\end{lstlisting}
\end{genpart}
\end{rolloutbox}


\subsection{Samples from Qwen3-8B-Base (AR)}

\subsubsection{Step 10}
\begin{rolloutbox}{Qwen3-8B-Base - Step 10}
\begin{promptpart}
\begin{lstlisting}
<|im_start|>system
You are a helpful assistant. You first think about the reasoning process in your mind and then provide the user with the answer.<|im_end|>
<|im_start|>user
Using the numbers [84, 4, 5], create an equation that equals 64. You can use basic arithmetic operations (+, -, *, /) and each number can only be used once. Show your work in <think> </think> tags. And return the final answer in <answer> </answer> tags, for example <answer> (1 + 2) / 3 </answer>.<|im_end|>
<|im_start|>assistant
Let me solve this step by step.
<think>
\end{lstlisting}
\end{promptpart}

\begin{genpart}
\begin{lstlisting}
### Reasoning Process:

To create an equation that equals 64 using the numbers [84, 4, 5] with basic arithmetic operations, I need to explore different combinations of these numbers and operations. Here's how I approached it:

1. **Identify the target**: The equation should equal 64.
2. **List the numbers**: 84, 4, 5.
3. **Explore operations**: I considered addition (+), subtraction (-), multiplication (*), and division (/).
4. **Try combinations**:
   - **Multiplication and Subtraction**: I thought about multiplying two numbers and then subtracting the third.
   - **Division and Addition**: I also considered dividing one number by another and then adding the third.
5. **Find the correct combination**:
   - I tried \\( 84 - (4 \\times 5) \\):
     - \\( 4 \\times 5 = 20 \\)
     - \\( 84 - 20 = 64 \\)
   - This combination works!

### Final Answer:

<answer>84 - (4 * 5)</answer><|endoftext|>
\end{lstlisting}
\end{genpart}
\end{rolloutbox}

\subsubsection{Step 300}
\begin{rolloutbox}{Qwen3-8B-Base - Step 300}
\begin{promptpart}
\begin{lstlisting}
<|im_start|>system
You are a helpful assistant. You first think about the reasoning process in your mind and then provide the user with the answer.<|im_end|>
<|im_start|>user
Using the numbers [68, 56, 82, 29], create an equation that equals 65. You can use basic arithmetic operations (+, -, *, /) and each number can only be used once. Show your work in <think> </think> tags. And return the final answer in <answer> </answer> tags, for example <answer> (1 + 2) / 3 </answer>.<|im_end|>
<|im_start|>assistant
Let me solve this step by step.
<think>
\end{lstlisting}
\end{promptpart}

\begin{genpart}
\begin{lstlisting}
### Reasoning Process:

To create an equation that equals 65 using the numbers [68, 56, 82, 29] and basic arithmetic operations (+, -, *, /), I need to explore different combinations of these numbers and operations. Since each number can only be used once, I need to find a combination where the result of the operations equals 65.

### Step-by-Step Solution:

1. **Identify the numbers and target**:  
   Numbers: 68, 56, 82, 29  
   Target: 65  

2. **Explore possible combinations**:  
   - **Combination 1**:  
     \\( 82 + 56 - 29 - 68 \\)  
     \\( 138 - 29 - 68 = 41 \\) (Not 65)  

   - **Combination 2**:  
     \\( 82 + 29 - 56 + 68 \\)  
     \\( 111 - 56 + 68 = 123 \\) (Not 65)  

   - **Combination 3**:  
     \\( 82 - 56 + 68 - 29 \\)  
     \\( 26 + 68 - 29 = 65 \\) (This works!)  

3. **Verify the solution**:  
   \\( 82 - 56 + 68 - 29 = 65 \\)  

   - \\( 82 - 56 = 26 \\)  
   - \\( 26 + 68 = 94 \\)  
   - \\( 94 - 29 = 65 \\)  

   The equation is correct.  

### Final Answer:  
<answer>82 - 56 + 68 - 29</answer><|endoftext|>
\end{lstlisting}
\end{genpart}
\end{rolloutbox}

\subsubsection{Step 500}
\begin{rolloutbox}{Qwen3-8B-Base - Step 500}
\begin{promptpart}
\begin{lstlisting}
<|im_start|>system
You are a helpful assistant. You first think about the reasoning process in your mind and then provide the user with the answer.<|im_end|>
<|im_start|>user
Using the numbers [68, 56, 82, 29], create an equation that equals 65. You can use basic arithmetic operations (+, -, *, /) and each number can only be used once. Show your work in <think> </think> tags. And return the final answer in <answer> </answer> tags, for example <answer> (1 + 2) / 3 </answer>.<|im_end|>
<|im_start|>assistant
Let me solve this step by step.
<think>
\end{lstlisting}
\end{promptpart}

\begin{genpart}
\begin{lstlisting}
### Reasoning Process:

To create an equation that equals 65 using the numbers [68, 56, 82, 29] and basic arithmetic operations (+, -, *, /), I need to explore all possible combinations of these numbers and operations. Since each number can only be used once, I\u2019ll systematically test different arrangements.

### Step-by-Step Solution:

1. **Identify the numbers and target**:  
   Numbers: 68, 56, 82, 29  
   Target: 65  

2. **Explore possible combinations**:  
   - **Combination 1**: 82 - 68 + 56 - 29  
     Calculation: 82 - 68 = 14  
     14 + 56 = 70  
     70 - 29 = 41 (Not 65)  

   - **Combination 2**: 82 - 56 + 68 - 29  
     Calculation: 82 - 56 = 26  
     26 + 68 = 94  
     94 - 29 = 65 (This works!)  

3. **Verify the solution**:  
   The equation 82 - 56 + 68 - 29 equals 65, which matches the target.  

### Final Answer:  
<answer>82 - 56 + 68 - 29</answer><|endoftext|>
\end{lstlisting}
\end{genpart}
\end{rolloutbox}

\subsection{Samples from SDAR-8B-Chat (DLM)}

\subsubsection{Step 10}
\begin{rolloutbox}{SDAR-8B-Chat - Step 10}
\begin{promptpart}
\begin{lstlisting}
<|im_start|>system
You are a helpful assistant. You first think about the reasoning process in your mind and then provide the user with the answer.<|im_end|>
<|im_start|>user
Using the numbers [60, 89, 26, 43], create an equation that equals 91. You can use basic arithmetic operations (+, -, *, /) and each number can only be used once. Show your work in <think> </think> tags. And return the final answer in <answer> </answer> tags, for example <answer> (1 + 2) / 3 </answer>.<|im_end|>
<|im_start|>assistant
Let me solve this step by step.
<think>
\end{lstlisting}
\end{promptpart}

\begin{genpart}
\begin{lstlisting}
1. First, I will try to find a combination of numbers and operations that can give me 24.
2. I will start by trying to add the numbers together to see if I can get 24.
3. I will then try to subtract the numbers to see if I can get 24.
4. I will then try to multiply the numbers to see if I can get 24.
5. Finally, I will try to divide the numbers to see if I can get 24.
6. After trying all the combinations, I found that the equation (11 + 57) - (25 + 67) equals 24.
<answer>(11 + 57) - (25 + 67)</answer><|im_end|>
<|endoftext|><|endoftext|>
\end{lstlisting}
\end{genpart}
\end{rolloutbox}

\subsubsection{Step 300}
\begin{rolloutbox}{SDAR-8B-Chat - Step 300}
\begin{promptpart}
\begin{lstlisting}
<|im_start|>system
You are a helpful assistant. You first think about the reasoning process in your mind and then provide the user with the answer.<|im_end|>
<|im_start|>user
Using the numbers [26, 3, 22], create an equation that equals 51. You can use basic arithmetic operations (+, -, *, /) and each number can only be used once. Show your work in <think> </think> tags. And return the final answer in <answer> </answer> tags, for example <answer> (1 + 2) / 3 </answer>.<|im_end|>
<|im_start|>assistant
Let me solve this step by step.
<think>
\end{lstlisting}
\end{promptpart}

\begin{genpart}
\begin{lstlisting}
26 + 3 + 22 = 51
<answer>26 + 3 + 22</answer><|im_end|>
<|endoftext|><|endoftext|><|endoftext|><|endoftext|>
\end{lstlisting}
\end{genpart}
\end{rolloutbox}

\subsubsection{Step 500}
\begin{rolloutbox}{SDAR-8B-Chat - Step 500}
\begin{promptpart}
\begin{lstlisting}
<|im_start|>system
You are a helpful assistant. You first think about the reasoning process in your mind and then provide the user with the answer.<|im_end|>
<|im_start|>user
Using the numbers [68, 56, 82, 29], create an equation that equals 65. You can use basic arithmetic operations (+, -, *, /) and each number can only be used once. Show your work in <think> </think> tags. And return the final answer in <answer> </answer> tags, for example <answer> (1 + 2) / 3 </answer>.<|im_end|>
<|im_start|>assistant
Let me solve this step by step.
<think>
\end{lstlisting}
\end{promptpart}

\begin{genpart}
\begin{lstlisting}
68 - 56 = 12
12 + 82 = 94
94 - 29 = 65
<answer>68 - 56 + 82 - 29</answer><|im_end|>
<|endoftext|><|endoftext|><|endoftext|>
\end{lstlisting}
\end{genpart}
\end{rolloutbox}

\subsection{Samples from LLaDA-8B-Instruct (DLM)}

\subsubsection{Step 10}
\begin{rolloutbox}{LLaDA-8B-Instruct - Step 10}
\begin{promptpart}
\begin{lstlisting}
<|startoftext|><|start_header_id|>system<|end_header_id|>

You are a helpful assistant. You first think about the reasoning process in your mind and then provide the user with the answer.<|eot_id|><|start_header_id|>user<|end_header_id|>

Using the numbers [84, 4, 5], create an equation that equals 64. You can use basic arithmetic operations (+, -, *, /) and each number can only be used once. Show your work in <think> </think> tags. And return the final answer in <answer> </answer> tags, for example <answer> (1 + 2) / 3 </answer>.<|eot_id|><|start_header_id|>assistant<|end_header_id|>

Let me solve this step by step.
<think>
\end{lstlisting}
\end{promptpart}

\begin{genpart}
\begin{lstlisting}
We need to create an equation that equals 64 using the numbers 84, 4, and 5. We can start by trying to find a combination of these numbers that equals 64. One possible combination is 84 - 4 * 5, which equals 84 - 20 = 64. We can try using this combination with the other numbers to create an equation that equals 64.
</think>
<answer>84 - 4 * 5</answer><|eot_id|><|endoftext|>
\end{lstlisting}
\end{genpart}
\end{rolloutbox}

\subsubsection{Step 300}
\begin{rolloutbox}{LLaDA-8B-Instruct - Step 300}
\begin{promptpart}
\begin{lstlisting}
<|startoftext|><|start_header_id|>system<|end_header_id|>

You are a helpful assistant. You first think about the reasoning process in your mind and then provide the user with the answer.<|eot_id|><|start_header_id|>user<|end_header_id|>

Using the numbers [77, 88, 93], create an equation that equals 82. You can use basic arithmetic operations (+, -, *, /) and each number can only be used once. Show your work in <think> </think> tags. And return the final answer in <answer> </answer> tags, for example <answer> (1 + 2) / 3 </answer>.<|eot_id|><|start_header_id|>assistant<|end_header_id|>

Let me solve this step by step.
<think>
\end{lstlisting}
\end{promptpart}

\begin{genpart}
\begin{lstlisting}
First, let's list the numbers we have: [77, 88, 93]. We need to come up with an equation that equals 82. We can use the basic arithmetic operations (+, -, *, /). We can also use parentheses to change the order of the operations. Let's start by adding and subtracting the numbers in different combinations until we find an equation that equals 82.
</think>
<answer> 93 - 88 + 77 </answer><|eot_id|><|endoftext|>
\end{lstlisting}
\end{genpart}
\end{rolloutbox}

\subsubsection{Step 500}
\begin{rolloutbox}{LLaDA-8B-Instruct - Step 500}
\begin{promptpart}
\begin{lstlisting}
<|startoftext|><|start_header_id|>system<|end_header_id|>

You are a helpful assistant. You first think about the reasoning process in your mind and then provide the user with the answer.<|eot_id|><|start_header_id|>user<|end_header_id|>

Using the numbers [71, 9, 1], create an equation that equals 79. You can use basic arithmetic operations (+, -, *, /) and each number can only be used once. Show your work in <think> </think> tags. And return the final answer in <answer> </answer> tags, for example <answer> (1 + 2) / 3 </answer>.<|eot_id|><|start_header_id|>assistant<|end_header_id|>

Let me solve this step by step.
<think>
\end{lstlisting}
\end{promptpart}

\begin{genpart}
\begin{lstlisting}
equals equals equals equals equals equals equals equals equals equals equals equals equals equals equals equals equals equals equals equals equals equals equals equals equals equals equals equals equals equals equals equals equals equals equals equals equals equals equals equals equals equals equals equals equals equals equals equals equals equals equals equals equals equals equals equals equals equals equals equals equals equals equals equals equals equals equals equals equals equals equals equals equals equals equals equals equals equals equals equals equals equals equals equals equals equals equals equals equals equals equals equals </think>
<answer> 71 + 9 - 1 </answer><|eot_id|><|endoftext|>
\end{lstlisting}
\end{genpart}
\end{rolloutbox}

\section{Supplementary Experimental Results}
\label{sec:appendix_figs}

% --- AR Models Figure (4 Rows, 3 Columns) ---
\begin{figure*}[ht]
    \centering
    % Row 1: a, b, c
    \begin{subfigure}[b]{0.32\textwidth}
        \centering
        \includegraphics[width=\textwidth]{figures/ar_comparisons/train_finished_rate.pdf}
        \caption{Train Finished Rate}
    \end{subfigure}
    \hfill
    \begin{subfigure}[b]{0.32\textwidth}
        \centering
        \includegraphics[width=\textwidth]{figures/ar_comparisons/train_success_rate.pdf}
        \caption{Train Success Rate}
    \end{subfigure}
    \hfill
    \begin{subfigure}[b]{0.32\textwidth}
        \centering
        \includegraphics[width=\textwidth]{figures/ar_comparisons/train_finished_success.pdf}
        \caption{Train Finished Success}
    \end{subfigure} \\ \vspace{0.5em}

    % Row 2: d, e, f
    \begin{subfigure}[b]{0.32\textwidth}
        \centering
        \includegraphics[width=\textwidth]{figures/ar_comparisons/train_entropy.pdf}
        \caption{Train Entropy}
    \end{subfigure}
    \hfill
    \begin{subfigure}[b]{0.32\textwidth}
        \centering
        \includegraphics[width=\textwidth]{figures/ar_comparisons/train_loss.pdf}
        \caption{Train Loss}
    \end{subfigure}
    \hfill
    \begin{subfigure}[b]{0.32\textwidth}
        \centering
        \includegraphics[width=\textwidth]{figures/ar_comparisons/train_mean_response_length.pdf}
         \caption{Mean Response Length}
    \end{subfigure} \\ \vspace{0.5em}

    % Row 3: g, h, i
    \begin{subfigure}[b]{0.32\textwidth}
        \centering
        \includegraphics[width=\textwidth]{figures/ar_comparisons/train_format_reward.pdf}
        \caption{Format Reward}
    \end{subfigure}
    \hfill
    \begin{subfigure}[b]{0.32\textwidth}
        \centering
        \includegraphics[width=\textwidth]{figures/ar_comparisons/train_mean_reward.pdf}
        \caption{Mean Reward}
    \end{subfigure}
    \hfill
    \begin{subfigure}[b]{0.32\textwidth}
        \centering
        \includegraphics[width=\textwidth]{figures/ar_comparisons/train_std_reward.pdf}
        \caption{Reward Std Dev}
    \end{subfigure} \\ \vspace{0.5em}

    % Row 4: j, k, l
    \begin{subfigure}[b]{0.32\textwidth}
        \centering
        \includegraphics[width=\textwidth]{figures/ar_comparisons/eval_finished_rate.pdf}
        \caption{Eval Finished Rate}
    \end{subfigure}
    \hfill
    \begin{subfigure}[b]{0.32\textwidth}
        \centering
        \includegraphics[width=\textwidth]{figures/ar_comparisons/eval_success_rate.pdf}
        \caption{Eval Success Rate}
    \end{subfigure}
    \hfill
    \begin{subfigure}[b]{0.32\textwidth}
        \centering
        \includegraphics[width=\textwidth]{figures/ar_comparisons/eval_finished_success.pdf}
        \caption{Eval Finished Success}
    \end{subfigure}

    \caption{Training dynamics of Autoregressive models (Qwen2.5 and Qwen3) on the Countdown task. The results illustrate the emergence of reasoning capabilities directly from base models through on-policy RL, highlighting steady progress in both accuracy and response structuring.}
    \label{fig:all_metrics_ar}
\end{figure*}

% --- DLM Models Figure (4 Rows, 3 Columns) ---
\begin{figure*}[ht]
    \centering
    % Row 1: a, b, c
    \begin{subfigure}[b]{0.32\textwidth}
        \centering
        \includegraphics[width=\textwidth]{figures/dlm_comparisons/train_finished_rate.pdf}
        \caption{Train Finished Rate}
    \end{subfigure}
    \hfill
    \begin{subfigure}[b]{0.32\textwidth}
        \centering
        \includegraphics[width=\textwidth]{figures/dlm_comparisons/train_success_rate.pdf}
        \caption{Train Success Rate}
    \end{subfigure}
    \hfill
    \begin{subfigure}[b]{0.32\textwidth}
        \centering
        \includegraphics[width=\textwidth]{figures/dlm_comparisons/train_finished_success_rate.pdf}
        \caption{Train Finished Success}
    \end{subfigure} \\ \vspace{0.5em}

    % Row 2: d, e, f
    \begin{subfigure}[b]{0.32\textwidth}
        \centering
        \includegraphics[width=\textwidth]{figures/dlm_comparisons/train_entropy.pdf}
        \caption{Train Entropy}
    \end{subfigure}
    \hfill
    \begin{subfigure}[b]{0.32\textwidth}
        \centering
        \includegraphics[width=\textwidth]{figures/dlm_comparisons/train_loss.pdf}
        \caption{Train Loss}
    \end{subfigure}
    \hfill
    \begin{subfigure}[b]{0.32\textwidth}
        \centering
        \includegraphics[width=\textwidth]{figures/dlm_comparisons/train_mean_response_len.pdf}
         \caption{Mean Response Length}
    \end{subfigure} \\ \vspace{0.5em}

    % Row 3: g, h, i
    \begin{subfigure}[b]{0.32\textwidth}
        \centering
        \includegraphics[width=\textwidth]{figures/dlm_comparisons/train_format_reward.pdf}
        \caption{Format Reward}
    \end{subfigure}
    \hfill
    \begin{subfigure}[b]{0.32\textwidth}
        \centering
        \includegraphics[width=\textwidth]{figures/dlm_comparisons/train_mean_reward.pdf}
        \caption{Mean Reward}
    \end{subfigure}
    \hfill
    \begin{subfigure}[b]{0.32\textwidth}
        \centering
        \includegraphics[width=\textwidth]{figures/dlm_comparisons/train_std_reward.pdf}
        \caption{Reward Std Dev}
    \end{subfigure} \\ \vspace{0.5em}

    % Row 4: j, k, l
    \begin{subfigure}[b]{0.32\textwidth}
        \centering
        \includegraphics[width=\textwidth]{figures/dlm_comparisons/eval_finished_rate.pdf}
        \caption{Eval Finished Rate}
    \end{subfigure}
    \hfill
    \begin{subfigure}[b]{0.32\textwidth}
        \centering
        \includegraphics[width=\textwidth]{figures/dlm_comparisons/eval_success_rate.pdf}
        \caption{Eval Success Rate}
    \end{subfigure}
    \hfill
    \begin{subfigure}[b]{0.32\textwidth}
        \centering
        \includegraphics[width=\textwidth]{figures/dlm_comparisons/eval_finished_success_rate.pdf}
        \caption{Eval Finished Success}
    \end{subfigure}

    \caption{Performance metrics for Diffusion Language Models (LLaDA and SDAR) during DGRPO training. Note the rapid decline in entropy across iterations, which characterizes the mode-seeking behavior and stability challenges inherent in diffusion-based reinforcement learning.}
    \label{fig:all_metrics_dlm}
\end{figure*}


\end{document}
