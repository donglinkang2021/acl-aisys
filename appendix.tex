\section{Rollout Generation Samples}
\label{appendix:rollout_samples}
This appendix provides samples of the generated text (rollouts) from different models at various stages of the reinforcement learning process.

\subsection{Samples from Qwen3-0.6B-Base (AR)}

\subsubsection{Step 10}
\begin{rolloutbox}{Qwen3-0.6B-Base - Step 10}
\begin{promptpart}
\begin{lstlisting}
<|im_start|>system
You are a helpful assistant. You first think about the reasoning process in your mind and then provide the user with the answer.<|im_end|>
<|im_start|>user
Using the numbers [43, 23, 14], create an equation that equals 52. You can use basic arithmetic operations (+, -, *, /) and each number can only be used once. Show your work in <think> </think> tags. And return the final answer in <answer> </answer> tags, for example <answer> (1 + 2) / 3 </answer>.<|im_end|>
<|im_start|>assistant
Let me solve this step by step.
<think>
\end{lstlisting}
\end{promptpart}

\begin{genpart}
\begin{lstlisting}
First, let's identify the numbers: 43, 23, and 14.
itemBuilder
I will combine these numbers in a way that their sum equals 52.
itemBuilder
Here\u2019s how I plan to achieve this:
itemBuilder
1. Add 43 and 23: 43 + 23 = 66
itemBuilder
2. Subtract 14 from the result: 66 - 14 = 52
itemBuilder
So, the equation that equals 52 is: 43 + 23 - 14 = 52
itemBuilder
Therefore, the final answer is: <answer> (43 + 23) - 14 </answer><|endoftext|>
\end{lstlisting}
\end{genpart}
\end{rolloutbox}

\subsubsection{Step 300}
\begin{rolloutbox}{Qwen3-0.6B-Base - Step 300}
\begin{promptpart}
\begin{lstlisting}
<|im_start|>system
You are a helpful assistant. You first think about the reasoning process in your mind and then provide the user with the answer.<|im_end|>
<|im_start|>user
Using the numbers [68, 56, 82, 29], create an equation that equals 65. You can use basic arithmetic operations (+, -, *, /) and each number can only be used once. Show your work in <think> </think> tags. And return the final answer in <answer> </answer> tags, for example <answer> (1 + 2) / 3 </answer>.<|im_end|>
<|im_start|>assistant
Let me solve this step by step.
<think>
\end{lstlisting}
\end{promptpart}

\begin{genpart}
\begin{lstlisting}
\u6218\u573a\u4e0a Blocks\u5171\u8ba1\\S\u52a8\u751f\u6210\u2bdb\ufffd calor\u52a0\u76d6\u4ea4\u7ed9 amount :
1. We have the numbers: 68, 56, 82, 29.
2. We need to find an equation that equals 65.
3. Let's try different combinations of these numbers.
4. One possible combination is: 82 - 68 + 56 - 29.
5. Calculate inside the parentheses: 82 - 68 = 14.
6. Then, 14 + 56 = 70.
7. Finally, 70 - 29 = 41.
8. This doesn't equal 65.
9. Let's try another combination: 82 - 56 + 68 - 29.
10. Calculate inside the parentheses: 82 - 56 = 26.
11. Then, 26 + 68 = 94.
12. Finally, 94 - 29 = 65.
13. This equals 65.
14. The equation is: 82 - 56 + 68 - 29.
15. Calculate inside the parentheses: 82 - 56 = 26.
16. Then, 26 + 68 = 94.
17. Finally, 94 - 29 = 65.
18. The equation is: 82 - 56 + 68 - 29.
19. The final answer is: <answer> (82 - 56) + (68 - 29) </answer>.<|endoftext|>
\end{lstlisting}
\end{genpart}
\end{rolloutbox}

\subsubsection{Step 500}
\begin{rolloutbox}{Qwen3-0.6B-Base - Step 500}
\begin{promptpart}
\begin{lstlisting}
<|im_start|>system
You are a helpful assistant. You first think about the reasoning process in your mind and then provide the user with the answer.<|im_end|>
<|im_start|>user
Using the numbers [68, 56, 82, 29], create an equation that equals 65. You can use basic arithmetic operations (+, -, *, /) and each number can only be used once. Show your work in <think> </think> tags. And return the final answer in <answer> </answer> tags, for example <answer> (1 + 2) / 3 </answer>.<|im_end|>
<|im_start|>assistant
Let me solve this step by step.
<think>
\end{lstlisting}
\end{promptpart}

\begin{genpart}
\begin{lstlisting}
 Tap\u0c9b_H\ufffd.cookies =>
1. We have the numbers: 68, 56, 82, 29.
2. We need to find an equation that equals 65.
3. Let's try different combinations of these numbers.
4. One possible combination is: 82 - 68 + 56 - 29.
5. Calculate inside the parentheses: 82 - 68 = 14.
6. Then, 14 + 56 = 70.
7. Finally, 70 - 29 = 41.
8. This doesn't equal 65.
9. Let's try another combination: 82 - 56 + 68 - 29.
10. Calculate inside the parentheses: 82 - 56 = 26.
11. Then, 26 + 68 = 94.
12. Finally, 94 - 29 = 65.
13. This equals 65.
14. The equation is: 82 - 56 + 68 - 29.
15. Calculate inside the parentheses: 82 - 56 = 26.
16. Then, 26 + 68 = 94.
17. Finally, 94 - 29 = 65.
18. The equation is: 82 - 56 + 68 - 29.
19. The final answer is: <answer> (82 - 56) + (68 - 29) </answer>.<|endoftext|>
\end{lstlisting}
\end{genpart}
\end{rolloutbox}


\subsection{Samples from Qwen3-8B-Base (AR)}

\subsubsection{Step 10}
\begin{rolloutbox}{Qwen3-8B-Base - Step 10}
\begin{promptpart}
\begin{lstlisting}
<|im_start|>system
You are a helpful assistant. You first think about the reasoning process in your mind and then provide the user with the answer.<|im_end|>
<|im_start|>user
Using the numbers [84, 4, 5], create an equation that equals 64. You can use basic arithmetic operations (+, -, *, /) and each number can only be used once. Show your work in <think> </think> tags. And return the final answer in <answer> </answer> tags, for example <answer> (1 + 2) / 3 </answer>.<|im_end|>
<|im_start|>assistant
Let me solve this step by step.
<think>
\end{lstlisting}
\end{promptpart}

\begin{genpart}
\begin{lstlisting}
### Reasoning Process:

To create an equation that equals 64 using the numbers [84, 4, 5] with basic arithmetic operations, I need to explore different combinations of these numbers and operations. Here's how I approached it:

1. **Identify the target**: The equation should equal 64.
2. **List the numbers**: 84, 4, 5.
3. **Explore operations**: I considered addition (+), subtraction (-), multiplication (*), and division (/).
4. **Try combinations**:
   - **Multiplication and Subtraction**: I thought about multiplying two numbers and then subtracting the third.
   - **Division and Addition**: I also considered dividing one number by another and then adding the third.
5. **Find the correct combination**:
   - I tried \\( 84 - (4 \\times 5) \\):
     - \\( 4 \\times 5 = 20 \\)
     - \\( 84 - 20 = 64 \\)
   - This combination works!

### Final Answer:

<answer>84 - (4 * 5)</answer><|endoftext|>
\end{lstlisting}
\end{genpart}
\end{rolloutbox}

\subsubsection{Step 300}
\begin{rolloutbox}{Qwen3-8B-Base - Step 300}
\begin{promptpart}
\begin{lstlisting}
<|im_start|>system
You are a helpful assistant. You first think about the reasoning process in your mind and then provide the user with the answer.<|im_end|>
<|im_start|>user
Using the numbers [68, 56, 82, 29], create an equation that equals 65. You can use basic arithmetic operations (+, -, *, /) and each number can only be used once. Show your work in <think> </think> tags. And return the final answer in <answer> </answer> tags, for example <answer> (1 + 2) / 3 </answer>.<|im_end|>
<|im_start|>assistant
Let me solve this step by step.
<think>
\end{lstlisting}
\end{promptpart}

\begin{genpart}
\begin{lstlisting}
### Reasoning Process:

To create an equation that equals 65 using the numbers [68, 56, 82, 29] and basic arithmetic operations (+, -, *, /), I need to explore different combinations of these numbers and operations. Since each number can only be used once, I need to find a combination where the result of the operations equals 65.

### Step-by-Step Solution:

1. **Identify the numbers and target**:  
   Numbers: 68, 56, 82, 29  
   Target: 65  

2. **Explore possible combinations**:  
   - **Combination 1**:  
     \\( 82 + 56 - 29 - 68 \\)  
     \\( 138 - 29 - 68 = 41 \\) (Not 65)  

   - **Combination 2**:  
     \\( 82 + 29 - 56 + 68 \\)  
     \\( 111 - 56 + 68 = 123 \\) (Not 65)  

   - **Combination 3**:  
     \\( 82 - 56 + 68 - 29 \\)  
     \\( 26 + 68 - 29 = 65 \\) (This works!)  

3. **Verify the solution**:  
   \\( 82 - 56 + 68 - 29 = 65 \\)  

   - \\( 82 - 56 = 26 \\)  
   - \\( 26 + 68 = 94 \\)  
   - \\( 94 - 29 = 65 \\)  

   The equation is correct.  

### Final Answer:  
<answer>82 - 56 + 68 - 29</answer><|endoftext|>
\end{lstlisting}
\end{genpart}
\end{rolloutbox}

\subsubsection{Step 500}
\begin{rolloutbox}{Qwen3-8B-Base - Step 500}
\begin{promptpart}
\begin{lstlisting}
<|im_start|>system
You are a helpful assistant. You first think about the reasoning process in your mind and then provide the user with the answer.<|im_end|>
<|im_start|>user
Using the numbers [68, 56, 82, 29], create an equation that equals 65. You can use basic arithmetic operations (+, -, *, /) and each number can only be used once. Show your work in <think> </think> tags. And return the final answer in <answer> </answer> tags, for example <answer> (1 + 2) / 3 </answer>.<|im_end|>
<|im_start|>assistant
Let me solve this step by step.
<think>
\end{lstlisting}
\end{promptpart}

\begin{genpart}
\begin{lstlisting}
### Reasoning Process:

To create an equation that equals 65 using the numbers [68, 56, 82, 29] and basic arithmetic operations (+, -, *, /), I need to explore all possible combinations of these numbers and operations. Since each number can only be used once, I\u2019ll systematically test different arrangements.

### Step-by-Step Solution:

1. **Identify the numbers and target**:  
   Numbers: 68, 56, 82, 29  
   Target: 65  

2. **Explore possible combinations**:  
   - **Combination 1**: 82 - 68 + 56 - 29  
     Calculation: 82 - 68 = 14  
     14 + 56 = 70  
     70 - 29 = 41 (Not 65)  

   - **Combination 2**: 82 - 56 + 68 - 29  
     Calculation: 82 - 56 = 26  
     26 + 68 = 94  
     94 - 29 = 65 (This works!)  

3. **Verify the solution**:  
   The equation 82 - 56 + 68 - 29 equals 65, which matches the target.  

### Final Answer:  
<answer>82 - 56 + 68 - 29</answer><|endoftext|>
\end{lstlisting}
\end{genpart}
\end{rolloutbox}

\subsection{Samples from SDAR-8B-Chat (DLM)}

\subsubsection{Step 10}
\begin{rolloutbox}{SDAR-8B-Chat - Step 10}
\begin{promptpart}
\begin{lstlisting}
<|im_start|>system
You are a helpful assistant. You first think about the reasoning process in your mind and then provide the user with the answer.<|im_end|>
<|im_start|>user
Using the numbers [60, 89, 26, 43], create an equation that equals 91. You can use basic arithmetic operations (+, -, *, /) and each number can only be used once. Show your work in <think> </think> tags. And return the final answer in <answer> </answer> tags, for example <answer> (1 + 2) / 3 </answer>.<|im_end|>
<|im_start|>assistant
Let me solve this step by step.
<think>
\end{lstlisting}
\end{promptpart}

\begin{genpart}
\begin{lstlisting}
1. First, I will try to find a combination of numbers and operations that can give me 24.
2. I will start by trying to add the numbers together to see if I can get 24.
3. I will then try to subtract the numbers to see if I can get 24.
4. I will then try to multiply the numbers to see if I can get 24.
5. Finally, I will try to divide the numbers to see if I can get 24.
6. After trying all the combinations, I found that the equation (11 + 57) - (25 + 67) equals 24.
<answer>(11 + 57) - (25 + 67)</answer><|im_end|>
<|endoftext|><|endoftext|>
\end{lstlisting}
\end{genpart}
\end{rolloutbox}

\subsubsection{Step 300}
\begin{rolloutbox}{SDAR-8B-Chat - Step 300}
\begin{promptpart}
\begin{lstlisting}
<|im_start|>system
You are a helpful assistant. You first think about the reasoning process in your mind and then provide the user with the answer.<|im_end|>
<|im_start|>user
Using the numbers [26, 3, 22], create an equation that equals 51. You can use basic arithmetic operations (+, -, *, /) and each number can only be used once. Show your work in <think> </think> tags. And return the final answer in <answer> </answer> tags, for example <answer> (1 + 2) / 3 </answer>.<|im_end|>
<|im_start|>assistant
Let me solve this step by step.
<think>
\end{lstlisting}
\end{promptpart}

\begin{genpart}
\begin{lstlisting}
26 + 3 + 22 = 51
<answer>26 + 3 + 22</answer><|im_end|>
<|endoftext|><|endoftext|><|endoftext|><|endoftext|>
\end{lstlisting}
\end{genpart}
\end{rolloutbox}

\subsubsection{Step 500}
\begin{rolloutbox}{SDAR-8B-Chat - Step 500}
\begin{promptpart}
\begin{lstlisting}
<|im_start|>system
You are a helpful assistant. You first think about the reasoning process in your mind and then provide the user with the answer.<|im_end|>
<|im_start|>user
Using the numbers [68, 56, 82, 29], create an equation that equals 65. You can use basic arithmetic operations (+, -, *, /) and each number can only be used once. Show your work in <think> </think> tags. And return the final answer in <answer> </answer> tags, for example <answer> (1 + 2) / 3 </answer>.<|im_end|>
<|im_start|>assistant
Let me solve this step by step.
<think>
\end{lstlisting}
\end{promptpart}

\begin{genpart}
\begin{lstlisting}
68 - 56 = 12
12 + 82 = 94
94 - 29 = 65
<answer>68 - 56 + 82 - 29</answer><|im_end|>
<|endoftext|><|endoftext|><|endoftext|>
\end{lstlisting}
\end{genpart}
\end{rolloutbox}

\subsection{Samples from LLaDA-8B-Instruct (DLM)}

\subsubsection{Step 10}
\begin{rolloutbox}{LLaDA-8B-Instruct - Step 10}
\begin{promptpart}
\begin{lstlisting}
<|startoftext|><|start_header_id|>system<|end_header_id|>

You are a helpful assistant. You first think about the reasoning process in your mind and then provide the user with the answer.<|eot_id|><|start_header_id|>user<|end_header_id|>

Using the numbers [84, 4, 5], create an equation that equals 64. You can use basic arithmetic operations (+, -, *, /) and each number can only be used once. Show your work in <think> </think> tags. And return the final answer in <answer> </answer> tags, for example <answer> (1 + 2) / 3 </answer>.<|eot_id|><|start_header_id|>assistant<|end_header_id|>

Let me solve this step by step.
<think>
\end{lstlisting}
\end{promptpart}

\begin{genpart}
\begin{lstlisting}
We need to create an equation that equals 64 using the numbers 84, 4, and 5. We can start by trying to find a combination of these numbers that equals 64. One possible combination is 84 - 4 * 5, which equals 84 - 20 = 64. We can try using this combination with the other numbers to create an equation that equals 64.
</think>
<answer>84 - 4 * 5</answer><|eot_id|><|endoftext|>
\end{lstlisting}
\end{genpart}
\end{rolloutbox}

\subsubsection{Step 300}
\begin{rolloutbox}{LLaDA-8B-Instruct - Step 300}
\begin{promptpart}
\begin{lstlisting}
<|startoftext|><|start_header_id|>system<|end_header_id|>

You are a helpful assistant. You first think about the reasoning process in your mind and then provide the user with the answer.<|eot_id|><|start_header_id|>user<|end_header_id|>

Using the numbers [77, 88, 93], create an equation that equals 82. You can use basic arithmetic operations (+, -, *, /) and each number can only be used once. Show your work in <think> </think> tags. And return the final answer in <answer> </answer> tags, for example <answer> (1 + 2) / 3 </answer>.<|eot_id|><|start_header_id|>assistant<|end_header_id|>

Let me solve this step by step.
<think>
\end{lstlisting}
\end{promptpart}

\begin{genpart}
\begin{lstlisting}
First, let's list the numbers we have: [77, 88, 93]. We need to come up with an equation that equals 82. We can use the basic arithmetic operations (+, -, *, /). We can also use parentheses to change the order of the operations. Let's start by adding and subtracting the numbers in different combinations until we find an equation that equals 82.
</think>
<answer> 93 - 88 + 77 </answer><|eot_id|><|endoftext|>
\end{lstlisting}
\end{genpart}
\end{rolloutbox}

\subsubsection{Step 500}
\begin{rolloutbox}{LLaDA-8B-Instruct - Step 500}
\begin{promptpart}
\begin{lstlisting}
<|startoftext|><|start_header_id|>system<|end_header_id|>

You are a helpful assistant. You first think about the reasoning process in your mind and then provide the user with the answer.<|eot_id|><|start_header_id|>user<|end_header_id|>

Using the numbers [71, 9, 1], create an equation that equals 79. You can use basic arithmetic operations (+, -, *, /) and each number can only be used once. Show your work in <think> </think> tags. And return the final answer in <answer> </answer> tags, for example <answer> (1 + 2) / 3 </answer>.<|eot_id|><|start_header_id|>assistant<|end_header_id|>

Let me solve this step by step.
<think>
\end{lstlisting}
\end{promptpart}

\begin{genpart}
\begin{lstlisting}
equals equals equals equals equals equals equals equals equals equals equals equals equals equals equals equals equals equals equals equals equals equals equals equals equals equals equals equals equals equals equals equals equals equals equals equals equals equals equals equals equals equals equals equals equals equals equals equals equals equals equals equals equals equals equals equals equals equals equals equals equals equals equals equals equals equals equals equals equals equals equals equals equals equals equals equals equals equals equals equals equals equals equals equals equals equals equals equals equals equals equals equals </think>
<answer> 71 + 9 - 1 </answer><|eot_id|><|endoftext|>
\end{lstlisting}
\end{genpart}
\end{rolloutbox}

\section{Supplementary Experimental Results}
\label{sec:appendix_figs}

% --- AR Models Figure (4 Rows, 3 Columns) ---
\begin{figure*}[ht]
    \centering
    % Row 1: a, b, c
    \begin{subfigure}[b]{0.32\textwidth}
        \centering
        \includegraphics[width=\textwidth]{figures/ar_comparisons/train_finished_rate.pdf}
        \caption{Train Finished Rate}
    \end{subfigure}
    \hfill
    \begin{subfigure}[b]{0.32\textwidth}
        \centering
        \includegraphics[width=\textwidth]{figures/ar_comparisons/train_success_rate.pdf}
        \caption{Train Success Rate}
    \end{subfigure}
    \hfill
    \begin{subfigure}[b]{0.32\textwidth}
        \centering
        \includegraphics[width=\textwidth]{figures/ar_comparisons/train_finished_success.pdf}
        \caption{Train Finished Success}
    \end{subfigure} \\ \vspace{0.5em}

    % Row 2: d, e, f
    \begin{subfigure}[b]{0.32\textwidth}
        \centering
        \includegraphics[width=\textwidth]{figures/ar_comparisons/train_entropy.pdf}
        \caption{Train Entropy}
    \end{subfigure}
    \hfill
    \begin{subfigure}[b]{0.32\textwidth}
        \centering
        \includegraphics[width=\textwidth]{figures/ar_comparisons/train_loss.pdf}
        \caption{Train Loss}
    \end{subfigure}
    \hfill
    \begin{subfigure}[b]{0.32\textwidth}
        \centering
        \includegraphics[width=\textwidth]{figures/ar_comparisons/train_mean_response_length.pdf}
         \caption{Mean Response Length}
    \end{subfigure} \\ \vspace{0.5em}

    % Row 3: g, h, i
    \begin{subfigure}[b]{0.32\textwidth}
        \centering
        \includegraphics[width=\textwidth]{figures/ar_comparisons/train_format_reward.pdf}
        \caption{Format Reward}
    \end{subfigure}
    \hfill
    \begin{subfigure}[b]{0.32\textwidth}
        \centering
        \includegraphics[width=\textwidth]{figures/ar_comparisons/train_mean_reward.pdf}
        \caption{Mean Reward}
    \end{subfigure}
    \hfill
    \begin{subfigure}[b]{0.32\textwidth}
        \centering
        \includegraphics[width=\textwidth]{figures/ar_comparisons/train_std_reward.pdf}
        \caption{Reward Std Dev}
    \end{subfigure} \\ \vspace{0.5em}

    % Row 4: j, k, l
    \begin{subfigure}[b]{0.32\textwidth}
        \centering
        \includegraphics[width=\textwidth]{figures/ar_comparisons/eval_finished_rate.pdf}
        \caption{Eval Finished Rate}
    \end{subfigure}
    \hfill
    \begin{subfigure}[b]{0.32\textwidth}
        \centering
        \includegraphics[width=\textwidth]{figures/ar_comparisons/eval_success_rate.pdf}
        \caption{Eval Success Rate}
    \end{subfigure}
    \hfill
    \begin{subfigure}[b]{0.32\textwidth}
        \centering
        \includegraphics[width=\textwidth]{figures/ar_comparisons/eval_finished_success.pdf}
        \caption{Eval Finished Success}
    \end{subfigure}

    \caption{Training dynamics of Autoregressive models (Qwen2.5 and Qwen3) on the Countdown task. The results illustrate the emergence of reasoning capabilities directly from base models through on-policy RL, highlighting steady progress in both accuracy and response structuring.}
    \label{fig:all_metrics_ar}
\end{figure*}

% --- DLM Models Figure (4 Rows, 3 Columns) ---
\begin{figure*}[ht]
    \centering
    % Row 1: a, b, c
    \begin{subfigure}[b]{0.32\textwidth}
        \centering
        \includegraphics[width=\textwidth]{figures/dlm_comparisons/train_finished_rate.pdf}
        \caption{Train Finished Rate}
    \end{subfigure}
    \hfill
    \begin{subfigure}[b]{0.32\textwidth}
        \centering
        \includegraphics[width=\textwidth]{figures/dlm_comparisons/train_success_rate.pdf}
        \caption{Train Success Rate}
    \end{subfigure}
    \hfill
    \begin{subfigure}[b]{0.32\textwidth}
        \centering
        \includegraphics[width=\textwidth]{figures/dlm_comparisons/train_finished_success_rate.pdf}
        \caption{Train Finished Success}
    \end{subfigure} \\ \vspace{0.5em}

    % Row 2: d, e, f
    \begin{subfigure}[b]{0.32\textwidth}
        \centering
        \includegraphics[width=\textwidth]{figures/dlm_comparisons/train_entropy.pdf}
        \caption{Train Entropy}
    \end{subfigure}
    \hfill
    \begin{subfigure}[b]{0.32\textwidth}
        \centering
        \includegraphics[width=\textwidth]{figures/dlm_comparisons/train_loss.pdf}
        \caption{Train Loss}
    \end{subfigure}
    \hfill
    \begin{subfigure}[b]{0.32\textwidth}
        \centering
        \includegraphics[width=\textwidth]{figures/dlm_comparisons/train_mean_response_len.pdf}
         \caption{Mean Response Length}
    \end{subfigure} \\ \vspace{0.5em}

    % Row 3: g, h, i
    \begin{subfigure}[b]{0.32\textwidth}
        \centering
        \includegraphics[width=\textwidth]{figures/dlm_comparisons/train_format_reward.pdf}
        \caption{Format Reward}
    \end{subfigure}
    \hfill
    \begin{subfigure}[b]{0.32\textwidth}
        \centering
        \includegraphics[width=\textwidth]{figures/dlm_comparisons/train_mean_reward.pdf}
        \caption{Mean Reward}
    \end{subfigure}
    \hfill
    \begin{subfigure}[b]{0.32\textwidth}
        \centering
        \includegraphics[width=\textwidth]{figures/dlm_comparisons/train_std_reward.pdf}
        \caption{Reward Std Dev}
    \end{subfigure} \\ \vspace{0.5em}

    % Row 4: j, k, l
    \begin{subfigure}[b]{0.32\textwidth}
        \centering
        \includegraphics[width=\textwidth]{figures/dlm_comparisons/eval_finished_rate.pdf}
        \caption{Eval Finished Rate}
    \end{subfigure}
    \hfill
    \begin{subfigure}[b]{0.32\textwidth}
        \centering
        \includegraphics[width=\textwidth]{figures/dlm_comparisons/eval_success_rate.pdf}
        \caption{Eval Success Rate}
    \end{subfigure}
    \hfill
    \begin{subfigure}[b]{0.32\textwidth}
        \centering
        \includegraphics[width=\textwidth]{figures/dlm_comparisons/eval_finished_success_rate.pdf}
        \caption{Eval Finished Success}
    \end{subfigure}

    \caption{Performance metrics for Diffusion Language Models (LLaDA and SDAR) during DGRPO training. Note the rapid decline in entropy across iterations, which characterizes the mode-seeking behavior and stability challenges inherent in diffusion-based reinforcement learning.}
    \label{fig:all_metrics_dlm}
\end{figure*}
